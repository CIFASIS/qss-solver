% $Header: /cvsroot/latex-beamer/latex-beamer/solutions/generic-talks/generic-ornate-15min-45min.en.tex,v 1.4 2004/10/07 20:53:08 tantau Exp $

\documentclass[10pt,pdflatex]{beamer}

\mode<presentation>
{
  \usetheme{Warsaw}
  \setbeamercovered{transparent}
}


%\usepackage[spanish]{babel}

\usepackage[latin1]{inputenc}
\usepackage{times}
\usepackage[T1]{fontenc}
\usepackage{ragged2e}
    \justifying

\setbeamertemplate{caption}[numbered]

\input{commands}


\everydisplay\expandafter{\the\everydisplay \color{blue}}
\everymath{\color{blue}}

\title{A Stand Alone Solver for Quantized State System Methods}
\author{Joaqu�n Fern�ndez\inst{1}\\ Ernesto Kofman \inst{1}} 
\institute[UNR] 
{\inst{1} FCEIA - Universidad Nacional de Rosario. CIFASIS - CONICET. Argentina}
\date{November 15, 2012}
\subject{Talks}

\AtBeginSection[]
{
  \begin{frame}<beamer>
    \frametitle{Outline}
    \begin{small}
    \tableofcontents[currentsection,currentsubsection]     
    \end{small}

  \end{frame}
}


\begin{document}

\begin{frame}
  \titlepage
\end{frame}


\begin{frame}
  \frametitle{Outline}
    \begin{small}
    \tableofcontents
    \end{small}
%  % You might wish to add the option [pausesections]
\end{frame}

\section{Quantized State System Methods} 

\begin{frame}{Motivating Example}
Consider the second order system:
\begin{equation}\label{eq:sosystem}
\begin{matrix}
\dot x_1(t)&=&x_2(t)\\
\dot x_2(t)&=&-x_1(t)
\end{matrix}
\end{equation}

and the following \alert{approximation}:
\begin{equation}\label{eq:qsosystem}
\begin{matrix}
\dot x_1(t)&=&\text{floor}(x_2(t))=q_2(t)\\
\dot x_2(t)&=&-\text{floor}(x_1(t))=-q_1(t)
\end{matrix}
\end{equation}
\end{frame}


\begin{frame} {Motivating Example}

\begin{columns}

\begin{column}{0.3\textwidth}
The \alert{quantized} equation
\begin{equation*}
\begin{matrix}
\dot x_1(t)&=&q_2(t)\\
\dot x_2(t)&=&-q_1(t)
\end{matrix}
\end{equation*}
can be easily solved. 

Consider, for instance, the i.v.
$x_1(0)=4.5$, $x_2(0)=0.5$:

\end{column}

\begin{column}{0.8\textwidth}
\begin{exampleblock}{}
\only<1>{\pdffile{qosc1}{7.8}}
\only<2>{\pdffile{qosc2}{7.8}}
\only<3>{\pdffile{qosc3}{7.8}}
\only<4->{\pdffile{qosc4}{7.8}}
\end{exampleblock}
\end{column}

\end{columns}

\end{frame}


\begin{frame}{Quantized State System Methods}
Given the system:
\begin{equation}\label{eq:ode}
\dot {\mathbf{x_a}}(t)=\mathbf{f}(\mathbf{x_a}(t),t)
\end{equation}
QSS methods approximate it by
\begin{equation}\label{eq:qss}
\dot {\mathbf{x}}(t)=\mathbf{f}(\mathbf{q}(t),t)
\end{equation}
Here, $\mathbf q(t)$ is the \alert{quantized state vector}. Its entries are componentwise related with those of the state vector $\mathbf x(t)$ by \alert{quantization functions}.

\begin{exampleblock}{}
 The different QSS methods differ in the usage of different quantization functions. 
\end{exampleblock}

\end{frame}

\begin{frame}{Quantized State System Methods: Features}
The different QSS methods share the following properties:
\begin{itemize}
 \item The state and quantized variables follow \alert{piecewise polynomial} trajectories.
 \item The methods are \alert{asynchronous}. Each quantized variable starts a new polynomial section at its own time.
 \item They satisfy nice \alert{stability} and \alert{error bound} properties.
 \item The resulting algorithms are efficient to simulate \alert{discontinuous systems}.
\end{itemize}

 
\end{frame}


\subsection{Different QSS Algorithms}

\begin{frame}{QSS1 Method}
 The first QSS method was the \alert{first order accurate} QSS1 \cite{KJ01,CK06}. It uses \alert{zero--order} hysteretic quantization (see Figure \ref{fig:ioquant}). 

In QSS1:
\begin{itemize}
 \item Quantized state trajectories are \alert{piecewise constant}.
 \item State trajectories are \alert{piecewise linear}.
 \item As it is only first order accurate, precise results require performing lots of steps.
 \item It is not suitable for simulating stiff systems (they provoke \alert{fast oscillations}).
\end{itemize}

\end{frame}

\begin{frame}{QSS1 Method -- Quantization Function}
\begin{figure}
 \begin{exampleblock}{}
  \pdffile{ioquant}{7}
 \end{exampleblock}
\caption{QSS1: Zero--order quantization}
\label{fig:ioquant}
\end{figure}
\end{frame}


\begin{frame}{QSS2 Method}
 A \alert{second order accurate} QSS method called QSS2 was proposed in \cite{Kof02a,CK06}. It uses \alert{first--order} hysteretic quantization (see Figure \ref{fig:ioquant2}). 

In QSS2:
\begin{itemize}
 \item Quantized state trajectories are \alert{piecewise linear}.
 \item State trajectories are \alert{piecewise parabolic}.
 \item Being second order accurate, it permits obtaining decent results performing less steps than QSS1.
 \item It is not suitable for simulating stiff systems.
\end{itemize}

\end{frame}

\begin{frame}{QSS2 Method -- Quantization Function}
\begin{figure}
 \begin{exampleblock}{}
  \pdffile{ioquant2}{7}
 \end{exampleblock}
\caption{QSS2: First--order quantization}
\label{fig:ioquant2}
\end{figure}
\end{frame}

\begin{frame}{QSS3 and QSS4 Methods}
 A \alert{third order accurate} QSS method called QSS3 was also developed. It uses \alert{second--order} hysteretic quantization (see Figure \ref{fig:ioquant3}). 

In QSS3:

\begin{itemize}
 \item Quantized state trajectories are \alert{piecewise parabolic}.
 \item State trajectories are \alert{piecewise cubic}.
 \item Being third order accurate, it permits improving the accuracy of QSS2 performing less steps.
 \item It is not suitable for simulating stiff systems.
\end{itemize}
Following these ideas, a fourth order QSS4 method was also developed.
\begin{alertblock}{}
 Since QSS methods are mainly intended for simulating discontinuous systems, higher order algorithms are not needed.
\end{alertblock}

\end{frame}

\begin{frame}{QSS3 Method -- Quantization Function}
\begin{figure}
 \begin{exampleblock}{}
  \pdffile{ioquant3}{7}
 \end{exampleblock}
\caption{QSS3: Second--order quantization}
\label{fig:ioquant3}
\end{figure}
\end{frame}


\begin{frame}{LIQSS Methods}
 Linearly Implicit versions of QSS methods were also developed \cite{MK09}. LIQSS methods have the following features:
\begin{itemize}
 \item They are \alert{explicit algorithms}, which are able to efficiently simulate many \alert{stiff systems}.
 \item There are methods of order 1 to 3 (LIQSS1, LIQSS2, and LIQSS3).
 \item Their cost per step is almost identical to that of QSS1, QSS2, and QSS3.
 \item They share the practical and theoretical properties of QSS methods.
 \item They are noticeably efficient to simulate circuits where switching components are modeled as large or small resistors (like in Modelica Standard Library).
 \item As they only look at the \alert{diagonal entries of the Jacobian matrix}, they cannot deal with stiff systems of arbitrary structure.
\end{itemize}

\end{frame}


\subsection{DEVS Implementation}

\begin{frame}{DEVS Implementation of QSS Methods}
Each component of the approximated ODE $ \dot x_i=f_i(q_1,\cdots,q_n,t)$ can be splitted 
in two equations: a static one:
\begin{equation} \label{eq:static}
 \dot x_i=f_i(q_1,\cdots,q_n,t)
\end{equation}
and a dynamic one:
\begin{equation}\label{eq:dynamic}
 q_i(t)=Q_i \left( \int_{t_0}^t \dot x_i(\tau) d\tau \right) 
\end{equation}
Both systems have piecewise polynomial input and output trajectories and can be easily represented as discrete event models. The DEVS models corresponding to Eq.\eqref{eq:static} are called \alert{static functions} while those corresponding to Eq.\eqref{eq:dynamic} are called \alert{quantized integrators}.    
\end{frame}

\begin{frame}{QSS Methods in DEVS -- Example}
 Consider the model of a bouncing ball:
\begin{equation} \label{eq:bball}
 \begin{split}
  \dot y &= v_y \\
  \dot v_y & = (-m\cdot g - F_f)/m\\  
 \end{split}
\end{equation}
where the floor force is 
\begin{equation} \label{eq:bball2}
 F_f=
\begin{cases}
 0 & \text{if~} y>0\\
 k\cdot y+b\cdot v_y& \text{otherwise}
\end{cases}
\end{equation}
\end{frame}


\begin{frame}{QSS Methods in PowerDEVS -- Examples}
\begin{figure}
 \jpgfile{pdevs_bball}{10}
  \caption{PowerDEVS \cite{BK11} Model of the Bouncing Ball System of Eqs.\eqref{eq:bball}--\eqref{eq:bball2}}  
\end{figure}

\end{frame}

\begin{frame}{QSS Methods in DEVS}
The DEVS implementation of QSS methods have the following features
 \begin{itemize}
  \item \alert{Static functions} are only evaluated after changes in the quantized state variables they explicitely depend on (or after discontinuity events).
  \item Discontinuities are \alert{locally handled} by the static functions.
  \item \alert{Zero crossing functions} are only evaluated after changes in the quantized state variables they explicitely depend on.
 \end{itemize}
\begin{exampleblock}{}
 The DEVS implementation exploits the knowledge of the \alert{full system structure} to evaluate functions only when they are needed.
\end{exampleblock}
\end{frame}

\begin{frame}{QSS Methods in DEVS}
 However, the DEVS implementation has some drawbacks:
\begin{itemize}
 \item It enforces the user to represent the models as Block Diagrams (although a prototype for transforming Modelica models into PowerDEVS Block Diagrams was developed).
 \item \alert{Static function evaluation is slow} as it involves some event traffic between different blocks.
 \item In complex models with many blocks, the \alert{DEVS scheduling mechanism} slows down further the simulations.
\end{itemize}
\begin{exampleblock}{}
 These drawbacks motivated the development of the \alert{Stand--Alone QSS Solver}.
\end{exampleblock}

\end{frame}



\subsection{Stand Alone Implementation: Pros and Cons}

\begin{frame}{Stand Alone Implementation: Pros and Cons}
A stand--alone QSS solver can have te following advantages:
 \begin{itemize}
  \item It can evaluate the state derivatives and zero crossing functions with a \alert{single function call}.
  \item Models are not limited to \alert{Block Diagram} descriptions.
  \item It will not suffer the \alert{overhead} imposed by the underlying \alert{DEVS simulation engine}.
 \end{itemize}
However, there are some disadvantages:
\begin{itemize}
 \item The models must be provided of \alert{structure information} so that functions are only evaluated when it is necessary.
 \item The evaluation of \alert{high order derivatives} becomes more difficult.
 \item The presence of \alert{input terms} is more difficult to handle. 
\end{itemize}
 
\end{frame}

\section{The Stand Alone QSS Solver}
\begin{frame}{The Stand Alone QSS Solver}
 We developed a \alert{Stand--Alone QSS Solver} with the following features:
\begin{itemize}
 \item It is entirely written in \alert{plain C} language.
 \item It implements QSS1, QSS2, QSS3, LIQSS1, LIQSS2, and LIQSS3 methods.
 \item It solves continuous and discontinuous ODEs.
\end{itemize}
Additionally, we have developed tools for:
\begin{itemize}
 \item Converting models defined in a subset of Modelica language (called \alert{MicroModelica}) into the C language description needed by the solver.
 \item Converting general Modelica models into the subset \alert{MicroModelica}.
 \item A simple Graphical User Interface to integrate the different utilities.
\end{itemize}

\end{frame}

\begin{frame}{Stand Alone QSS Solver -- Basic Scheme}
 \begin{figure}
  \pdffile{scheme}{9}
  \caption{Stand Alone QSS Solver -- Basic Scheme}
 \end{figure}

\end{frame}


\subsection{The Integrator}

\begin{frame}{The Integrator}
 The \alert{Integrator} is the \alert{core} of the solver. It is in charge of:
\begin{itemize}
 \item Integrating the state derivatives $\dot x_i(t)$ to obtain the \alert{state trajectories} $x_i(t)$.
 \item Advancing the \alert{simulation time}.
 \item Deciding when each state derivative and zero--crossing function should be re--evaluated.
 \item Searching for \alert{discontinuities}. 
 \item Invoking the model's discontinuity handlers when zero crossings are detected.
\end{itemize}
\end{frame}

\begin{frame}[fragile]{The Integrator -- Basic algorithm}
\begin{footnotesize}
 \begin{verbatim}
	Advance the simulation time until the next change time
	If the next change is a change in variable q_i then
	  Ask the quantizer the new value for q_i
	  Ask the quantizer when the next change in q_i will occur
	  Ask the model which state derivatives dx_j depend on q_i
	  for each dx_j depending on q_i
	      ask the model the new value for dx_j
	      ask the quantizer when the next change in q_j will occur
	  end for
	  Ask the model which zero crossing functions ZC_j depend on q_i
	  for each zc_j depending on q_i
	      ask the model the new value for ZC_j
	      compute the new crossing time for ZC_j
	  end for
 \end{verbatim}
 
\end{footnotesize}
\end{frame}

\begin{frame}[fragile]{The Integrator -- Basic algorithm (II)}
\begin{footnotesize}
 \begin{verbatim}

	else If the next change is a zero crossing in ZC_i
	  Tell the model to execute the handler H_i
	  Ask the model which state derivatives dx_j depend on changes at H_i
	  for each dx_j depending on H_i
	      ask the model the new value for dx_j
	      ask the quantizer when the next change in q_j will occur
	  end for
	  Ask the model which zero crossing functions ZC_j depend on H_i
	  for each zc_j depending on H_i
	      ask the model the new value for ZC_j
	      compute the new crossing time for ZC_j
	  end for
	  
 \end{verbatim}
 
\end{footnotesize}


\end{frame}

\subsection{The Quantizer}

\begin{frame}{The Quantizer}
 The \alert{quantizer} is in chage of computing the quantized states $q_i(t)$ according to the state $x_i(t)$ (and its derivatives) and the QSS method in use (QSS1, QSS2, QSS3, LIQSS1, LIQSS2, LIQSS3). It has the following functions:
\begin{itemize}
 \item Compute the new polynomial section for $q_i(t)$ as a function of $x_i(t)$.
 \item Calculate when a new polynomial section of $q_i(t)$ must start.
 \item Recalculate when a new polynomial section of $q_i(t)$ must start after a change in $\dot x_i(t)$.
\end{itemize}

\begin{exampleblock}{}
 The Quantizer and the Integrator constitute the \alert{QSS Solver}. The Solver is compiled together with the Model to obtain an executable simulation file.
\end{exampleblock}

\end{frame}

\subsection{The Models}
\begin{frame}{The Models}
 The models calculate the \alert{state derivatives} $\dot x_i(t)=f_i(q(t),t)$ and the zero-crossing functions $ZC_i(q(t),t)$. They also provide \alert{structure information} so the integrator knows which state derivatives and zero crossing function change after each step. 

The model functions are:
\begin{itemize}
 \item Evaluate a \alert{single state derivative} $\dot x_i=f_i(q,t)$.
 \item Evaluate all the state derivatives depending on one state $x_j$ in a single call.
 \item Evaluate a \alert{zero crossing function} $ZC_i(q,t)$.
 \item Execute a \alert{handler} $H_i(q,t)$.
 \item Provide the \alert{structure matrices} expressing the direct influence from states to derivatives, from states to zero crossing functions, from handlers to derivatives, and from handlers to zero crossing functions.  
\end{itemize}

\end{frame}



\subsection{The MicroModelica Parser and C Code Generator}

\begin{frame}{The MicroModelica Parser and C Code Generator}
\begin{alertblock}{}
 Compared to classic solvers, the QSS solver has the disadvantage that the model must provide additional information about the system structure, which is very uncomfortable for an end--user.
\end{alertblock}
  In order to solve this problem, a \alert{MicroModelica parser} and \alert{C Code Generator} was implemented with the following features:
\begin{itemize}
 \item It converts models written in a subset of Modelica language (\alert{MicroModelica}) into C coded models with all the functions and structure information required by the solver.
 \item It is optimized for large scale systems by preserving Modelica arrays in the C code generation.
 \item It allows the usage of external C functions.
 \item It has error control routines, providing useful debug information.
 \item Besides the C coded models, it generates the makefiles required for compiling the entire simulation.
\end{itemize}
\end{frame}

\section{Some Examples}
\subsection{A simple model of a Bouncing Ball}

\begin{frame}{A simple model of a Bouncing Ball}
\begin{itemize}
 \item  A possible MicroModelica representation of the bouncing ball model of Eq.\eqref{eq:bball} can be found in 
\url{./files/bball.mo}.

 \item The C code generated by the parser is in \url{./files/bball.c}.
\end{itemize}

\end{frame}

\subsection{Interleaved Buck Converter}

\begin{frame}{Interleaved Buck Converter}
 \begin{figure}
 \begin{exampleblock}{}
  \pdffile{interleaved}{11}  
 \end{exampleblock}
  \caption{Interleaved Buck Converter}
 \end{figure}

\end{frame}


\begin{frame}{Interleaved Buck Converter}
\begin{itemize}
 \item  A MicroModelica model of an \alert{interleaved Buck converter} can be found in 
\url{./files/interleaved.mo}.

 \item The C code generated by the parser is in \url{./files/interleaved.c}.

 \item Simulation Log files:
\begin{itemize}
\item OpenModelica: \url{./files/interleaved_omc.log}
\item Dymola: \url{./files/interleaved_dymola.log}
\item Stand Alone QSS solver \url{./files/interleaved.log}
\end{itemize}
\end{itemize}

\begin{exampleblock}{}
 In this case, the QSS solver with LIQSS2 algorithm is about 20 times faster than DASSL. Moreover, LIQSS2 results are more precise for the same accuracy settings.
\end{exampleblock}
 
\end{frame}

\subsection{Logical Inverter Chain}

\begin{frame}{A Chain of 500 Inverter}
The following model corresponds to a chain of 500 inverters
 \begin{equation}\label{eq:invmodel}
 \left\{
 \begin{array}{ll}
  \dot\omega_1(t)&=U_{op}-\omega_1(t)-\Upsilon g\left(u_{in}(t),\omega_1(t)\right)\\
  \dot\omega_j(t)&=U_{op}-\omega_j(t)-\Upsilon g\left(\omega_{j-1}(t),\omega_j(t)\right)\;\;\;\;j=2,3,..,500
 \end{array}
 \right.
\end{equation}
where
\begin{equation}\label{ch5ej3:gmax}
 g(u,v)=\left(max(u-U_{thres},0)\right)^2-\left(\max\left(u-v-U_{thres},0\right)\right)^2
\end{equation}
with the input term:
\begin{equation}
 u_{in}(t)=\left\{
 \begin{array}{ll}
  t-5&\text{if }5\leq t \leq 10\\
  5&\text{if }10< t \leq 15\\
  \frac{5}{2}(17-t)&\text{if }15< t \leq 17\\
  0&\text{otherwise}
 \end{array}
 \right.
\end{equation}

\end{frame}

\begin{frame}{A Chain of 500 Inverter}
 
\begin{itemize}
 \item  A MicroModelica model of an \alert{logical inverter chain} can be found in 
\url{./files/inverters.mo}.

 \item The C code generated by the parser is in \url{./files/inverters.c}.

 \item Simulation Log files:
\begin{itemize}
\item OpenModelica: \url{./files/inverters_omc.log}
\item Dymola: \url{./files/inverters_dymola.log}
\item Stand Alone QSS solver \url{./files/inverters.log}
\end{itemize}
\end{itemize}

\begin{exampleblock}{}
 In this example, the QSS solver with LIQSS2 algorithm is about 2000 times faster than DASSL. Compilation times are also noticeably faster due to \alert{array preservation} routines.
\end{exampleblock}
 

\end{frame}

\subsection{Interleaved Buck Converter Revisited}

\begin{frame}{Interleaved Buck Converter Revisited}
 \begin{figure}
  \jpgfile{interleaved_mod}{10}
  \caption{Interleaved Buck Converter with standard Modelica Components}
 \end{figure}

\end{frame}

\begin{frame}{Interleaved Buck Converter Revisited}
\begin{itemize}
 \item  The OpenModelica tool to convert from Modelica to MicroModelica \cite{BFBKC12} produces the following model \url{./files/interleaved_umo.mo} with additional external C functions for solving the algebraic loops \url{./files/interleaved_umo_external_functions.c}.
 \item Simulation log files:
 \begin{itemize}
\item OpenModelica: \url{./files/interleaved_umo_omc.log}
\item Dymola: \url{./files/interleaved_umo_dymola.log}
\item Stand Alone QSS solver \url{./files/interleaved_umo.log}
\end{itemize}
\end{itemize}

\begin{exampleblock}{}
 Now, LIQSS2 is more than 15 times faster than DASSL in Dymola and about 60 times faster than DASSL in OMC. Notice that Dymola produces better simulation code than OMC due to a more efficient \alert{tearing} algorithm.
\end{exampleblock}


\end{frame}



\section{Future Work and References}

\subsection{Future Work}
\begin{frame}{Future Work}
 \begin{itemize}
  \item We want to improve some features of the solver, particularly the evaluation of \alert{high order derivatives} (for QSS3 and LIQSS3) and the treatment of \alert{input signals}.  
  \item We have plans to specialize versions of our solver for some large--scale problems including \alert{Spiking neural networks} and \alert{MOL approximations of advection equations}.
  \item Another goal is to implement in the solver some recently developed \alert{parallel simulation} techniques for QSS methods \cite{BKC12}.
  \item We are currently working on improving the OpenModelica back--end to convert general Modelica models into MicroModelica. Once it is done, our goal is to include QSS methods in future OpenModelica releases.
 \end{itemize}

\end{frame}

\subsection{The Project}
\begin{frame}{The Project}
\begin{exampleblock}{}
The QSS Stand--Alone Solver is an \alert{Open Source} project developed by Joaqu�n Fern�ndez, Federico Bergero and Ernesto Kofman at the French--Argentine International Center for Information and Systems Sciences (CIFASIS -- CONICET, Rosario, Argentina).
\end{exampleblock}

 \begin{itemize}
 
 \item Project Website: \url{http://sourceforge.net/projects/qssengine/}.

 \item Ernesto Kofman's Website: \url{http://www.fceia.unr.edu.ar/~kofman/}
 \end{itemize}

\end{frame}

\subsection{References}

 \begin{frame}{References}
\begin{tiny}
  \bibliographystyle{alpha}
  \bibliography{journals,books,conferences}
 
\end{tiny}

 \end{frame}



\end{document}

